\documentclass{article}
\usepackage{amsmath}
\usepackage{fullpage}
\usepackage[]{geometry}
\usepackage{color}
\usepackage{graphicx}
\usepackage{physics}
\usepackage{showframe}
\usepackage{color}
\newcommand{\kt}{\mathbf{k}_{t}}
\newcommand{\Op}{\hat{\mathcal{\theta}}}
\newcommand{\n}{\hat{n}(\kt)}
\newcommand{\sa}{\hat{S}^{a}(\kt)}
\newcommand{\dw}{\delta(2\omega - \omega_{CV}(\kt))}


%--------------------Link colors and pdf info-----------------------%
\usepackage{hyperref}
\definecolor{links}{rgb}{0,0.2,0.6}
\hypersetup{colorlinks,breaklinks,urlcolor=links,linkcolor=links,pdftitle=Reinaldo A. Zapata Pe\~na - Curriculum Vitae (Complete),pdfauthor=Reinaldo A. Zapata Pe\~na}

\title{Optical Injection and Control of Spin and Electrical Currents}
\author{Reinaldo Arturo Zapata Pe\~na}
\date{}

\begin{document}
\maketitle

Math development from
\href{http://journals.aps.org/prb/abstract/10.1103/PhysRevB.68.165348}{PRB
\textbf{68}, 165348 (2003)}: All-optical control of spin and electrical currents
in quantum wells; Ali Najaime, R.D.R. Bath, J.E. Sipe.


\section{Quantum well states} % (fold)
\label{sec:quantum_well_states}

\subsection{Parameters and matrix elements}
{\small Section I-C in paper. \\}

Double degeneracy at each $\kt$. $\psi_{ns\kt}$ $\leftarrow$ spinor
wavefunctions: $n \equiv$ subband index; $s\equiv$ degeneracy index. Also $n$
and $m$ are used to refer to an arbitrary subband index with $s$ or $p$
degeneracy. $c$ is used for the conduction subbands and $v$ for the valence
subbands. $A$ corresponds to the area in the $xy$ plane.

At the end of the calculation $A \rightarrow \infty$ $\Rightarrow$
\begin{equation*}
\frac{1}{A} \sum_{\kt} \rightarrow \int \frac{d\kt}{(2 \pi)^{2}} .
\end{equation*}

Definitions:
\begin{align*}
v_{n,s;m,p}^{a}(\kt) = \langle ns\kt | v^{a} |
mp\kt \rangle & \leftarrow \text{velocity matrix elements}\\
S_{n,s;m,p}^{a}(\kt) = \langle ns\kt | S^{a} |
mp\kt \rangle & \leftarrow \text{spin matrix elements}\\
K_{n,s;m,p}^{ab}(\kt)= \langle ns\kt | v^{a}S^{b} |
mp\kt \rangle & \leftarrow \text{spin-velocity matrix elements}
\end{align*}
Superscripts denote Cartesian components and if repeated are to be summed over.

IPA, time recersal invariance leads to the following properties of the matrix
elements:
\begin{align*}
v_{n,\overline{s};m,\overline{p}}^{a*}(-\kt) &=
 -e^{i \lambda_{nm}} v_{n,s;m,p}^{a}(\kt)\\
S_{c,\overline{s};c,\overline{p}}^{a*}(-\kt) &=
 -e^{i \lambda'_{nm}} S_{c,s;c,p}^{a}(\kt)\\
K_{c,s;c,p}^{ab*}(\kt)&= 
 e^{i \lambda''_{nm}} K_{c,\overline{s};c,\overline{p}}^{ab}(-\kt)
\end{align*}
where $\overline{s}$ ($\overline{p}$) designates the opposite degeneracy index
of $s$ ($p$) and $\lambda$, $\lambda'$, and $\lambda''$ are arbitrary phases.
% section quantum_well_states (end)

\newpage
\section{Coherent control} % (fold)
\label{sec:coherent_control}
{\small Section II in paper \\}

For some experimental configuration the injected spin currents have an
accompanying electrical current. One can also generate \emph{pure} spin currents
where a sorting of optically injected process gives rise to electrons
propagating two opposite spatial directions with opposite average spins. The
direction of the propagation of the injected currents can be controlled via the
relative phase of the optical fields.

\subsection{A. Injection process}
{\small (Section II-A in paper.) \\}

Unperturbed Hamiltonian IPA second quantization:

\begin{equation*}\label{eq:unperturbed_hamiltonian}
H_{0} = \sum_{C\kt} \hbar \omega_{C}(\kt) \hat{a}^{\dag}_{C\kt} \hat{a}^{}_{C\kt}  - 
        \sum_{V\kt} \hbar \omega_{V}(\kt) \hat{b}^{\dag}_{C\kt} \hat{b}^{}_{C\kt}.
\end{equation*}

In a presence of an applied vector potential $\mathbf{A}(t)$ we have the
interaction Hamiltonian as
\begin{equation*}\label{eq:interaction_hamiltonian}
H_{int}(t) = - \frac{e}{c} \mathbf{A}(t) \cdot \mathcal{V}(t),
\end{equation*}
where
\begin{align*}
\mathcal{V}(t)  
=& \sum_{CC'\kt} \hat{a}^{\dag}_{C \kt} \hat{a}       _{C'\kt} 
\mathbf{v}_{CC'}(\kt) e^{i\omega_{CC'}(\kt)t} \\
-& \sum_{VV'\kt} \hat{b}^{\dag}_{V'\kt} \hat{b}       _{V \kt} 
\mathbf{v}_{VV'}(\kt) e^{i\omega_{VV'}(\kt)t} \\
+& \sum_{CV \kt} \hat{a}^{\dag}_{C \kt} \hat{b}^{\dag}_{V \kt} 
\mathbf{v}_{CV }(\kt) e^{i\omega_{CV }(\kt)t} \\
+& \sum_{CV \kt} \hat{b}       _{V \kt} \hat{a}       _{C \kt} 
\mathbf{v}_{VC }(\kt) e^{i\omega_{VC }(\kt)t},
\end{align*}
where $\hat{a}^{\dag}_{C\kt} $ ($\hat{a}_{C\kt}$) is the creation (annihilation)
operator for an electron with conduction subbands and spin indices denoted by
$c$ and $\hat{b}^{\dag}_{V\kt} $ ($\hat{b}_{V\kt}$) is the corresponding
creation (annihilation) operator for a hole with valence subbands and spin
indices denoted by $V$; $\hbar\omega_{CV}(\kt)$ is the energy of the conduction
and valence subbands $C/V$ at $\kt$

Vector potential of the field:
\begin{align*}\label{eq:vector_potential}
\mathbf{A}(t)   &=\mathbf{A}(  \omega)e^{-i(\omega+i\tau)t} 
                + \mathbf{A}(- \omega)e^{ i(\omega-i\tau)t} \\
                +&\mathbf{A}( 2\omega)e^{-i(\omega+i\tau)t} 
                + \mathbf{A}(-2\omega)e^{ i(\omega-i\tau)t},
\end{align*}
where $\tau$ is a small positive number to turn on the system at $t=-\infty$ and
is set to zero at the end of the calculation.

Electric field and vector potential relation:
\begin{equation*}\label{eq:electricF-vPotential_rel}
\mathbf{E} = \frac{i\omega}{c}\mathbf{A}(\omega).  
\end{equation*}

The optical field causes transitions from the ground state of the system,
$\ket{0}$, to the two particle (electron + hole) state:
\begin{equation*}\label{eq:transition}
\ket{CV\kt} = \hat{a}^{\dag}_{C\kt} \hat{a}_{V\kt} \ket{0}.
\end{equation*}

Perturbative approximation:
\begin{equation*}\label{eq:perturbative_approx}
\ket{\psi} = C_{0}(t) \ket{0} + \sum_{CV\kt} C_{CV\kt}(t)\ket{CV\kt}.
\end{equation*}
Solving for $C_{CV\kt}$ to second order in an electric field allows us to
calculate the rate of change of the expectation value of any single-particle
operator $\Op$ using the Fermi's golden rule (FGR, equation 6 of
the paper):
\begin{equation}\label{eq:FGR_1-2-I_contribution}
\frac{d \expval*{\Op}    }{dt} =
\frac{d \expval*{\Op}_{1}}{dt} +
\frac{d \expval*{\Op}_{2}}{dt} +
\frac{d \expval*{\Op}_{I}}{dt} ,
\end{equation}
where the subscripts 1, 2, and $I$ denotes one-photon contribution $2\omega$
process, two-photon $\omega$ process, and interference between excitation at 
$\omega$ and $2\omega$, respectively.

Each of the three components of Eq. \eqref{eq:FGR_1-2-I_contribution} will
include contributions from the conductions subbands $d\expval*{\Op}^{e}/dt$ and
valence subbands $d\expval*{\Op}^{h}/dt$ (equation 7 of the paper):
\begin{subequations}\label{eq:e-h_contribution_complete}
\begin{equation}\label{eq:e-h_contribution}
\frac{d\expval*{\Op}}{dt} =
\frac{d\expval*{\Op}^{e}}{dt} + 
\frac{d\expval*{\Op}^{h}}{dt} ,
\end{equation}
\begin{equation}\label{eq:e_contribution}
\frac{d\expval*{\Op}^{e}}{dt} = 
2\pi \sum_{C(C')V\kt} 
\Omega^{*}_{C'V}(\kt) 
\mel{C'\kt}{\Op}{C\kt}
\Omega_{CV}(\kt)
\delta\left( 2\omega-\omega_{CV}(\kt) \right),
\end{equation}
\begin{equation}\label{eq:h_contribution}
\frac{d\expval*{\Op}^{h}}{dt} = 
2\pi \sum_{V(V')C\kt} 
\Omega_{CV}(\kt)
\mel{V\kt}{\Op}{V'\kt}
\Omega^{*}_{CV'}(\kt)
\delta\left( 2\omega-\omega_{CV}(\kt) \right).
\end{equation}
\end{subequations}
Here the convention is $C=cs$, $C'=cs'$, $V=vs$, $V'=vs'$; so, the states differ
only in the spin index and then
\begin{equation*}\label{eq:omega_c-cp}
\omega_{CV}(\kt) = \omega_{C'V}(\kt).
\end{equation*}

The notation $C(C')$ indicates that the common subband index $c$ and the two
spin indices, $s$ and $s'$ are to be summed over. 

The term $\Omega_{CV}({\kt})$ in Eq. \eqref{eq:e-h_contribution_complete} is
given by (equation 8 in paper):
\begin{subequations}\label{eq:Omegas-complete}
\begin{equation}\label{eq:Omega_I-II}
\Omega_{CV}({\kt}) = \Omega^{I}_{CV}(\kt) + \Omega^{II}_{CV}(\kt),
\end{equation}
\begin{equation}\label{eq:Omega_I}
\Omega^{I}_{CV}(\kt) = \frac{ie}{\hbar}
\frac{\mathbf{v}_{CV}(\kt) \cdot \mathbf{E}(2\omega)}{2\omega}
\end{equation}
\begin{equation}\label{eq:Omega_II}
\Omega^{II}_{CV}(\kt) = -\left(\frac{e}{\hbar\omega}\right)^{2}
\sum_{N} \frac{[\mathbf{v}_{CN}(\kt) \cdot \mathbf{E}(\omega)]
[\mathbf{v}_{NV}(\kt)\cdot\mathbf{E}(\omega)]}{\omega_{CV}(\kt)+\omega_{VN}(kt)},
\end{equation}
\end{subequations}
where $\Omega^{I}(\kt)$ and $\Omega^{II}(\kt)$ correspond to one- and two-photon
transitions, respectively.

The hole subband have higher effective mass than the conduction subbands and so
the current is dominated by electrons. Then, substituting Eq.
\eqref{eq:e_contribution} into Eq. \eqref{eq:FGR_1-2-I_contribution} we have
that (equation 9 in paper)
\begin{subequations}\label{eq:electron_12I_contribution-complete}
\begin{equation}\label{eq:electron_12I_contribution}
\frac{d\expval*{\Op}^{e}}{dt} = 
\frac{d\expval*{\Op}^{e}_{1}}{dt} + 
\frac{d\expval*{\Op}^{e}_{2}}{dt} + 
\frac{d\expval*{\Op}^{e}_{I}}{dt} , 
\end{equation}
\begin{align}
\frac{d\expval*{\Op}^{e}_{1}}{dt} &= 
2\pi \sum_{C(C')V\kt}^n
\mel*{C'\kt}{\Op}{C\kt}
[\Omega^{I}_{CV}(\kt)
\Omega^{I*}_{C'V}(\kt)]
\delta \left( 2\omega - \omega_{CV}(\kt) \right) , 
\label{eq:1_electron_contribution} \\
\frac{d\expval*{\Op}^{e}_{2}}{dt} &= 
2\pi \sum_{C(C')V\kt}^n
\mel*{C'\kt}{\Op}{C\kt}
[\Omega^{II}_{CV}(\kt)
\Omega^{II*}_{C'V}(\kt)]
\delta \left( 2\omega - \omega_{CV}(\kt) \right) , 
\label{eq:2_electron_contribution}\\
\frac{d\expval*{\Op}^{e}_{I}}{dt} &= 
2\pi \sum_{C(C')V\kt}^n
\mel*{C'\kt}{\Op}{C\kt}
[\Omega^{I}_{CV}(\kt)
\Omega^{II*}_{C'V}(\kt) + c.c.]
\delta \left( 2\omega - \omega_{CV}(\kt) \right) 
\label{eq:electron_I_contribution}.
\end{align}
\end{subequations}

In previous equations the expression $\Op$ in can be substituted for the
electrical and spin current operators. Although the contributions from the holes
can be similarly found.

\subsection{Carrier and spin population}
{\small Section II-B in paper \\}

In all the following expressions literal superscripts denote Cartesian
components and if repeated are to be summed over.

Areal carrier density operator: 
\begin{align}
\hat{n} =& \frac{1}{\mathcal{A}} \sum_{\kt} \n , \nonumber \\ 
         &\n = \sum_{C} \hat{a}^{\dag}_{C\kt} 
           \hat{a}_{C\kt} , \nonumber \\ 
\hat{n} =& \frac{1}{\mathcal{A}} \sum_{C\kt} 
  \hat{a}^{\dag}_{C\kt} \hat{a}_{C\kt} . \label{eq:areal_carrier_density}
\end{align}

Areal $a$ component of the spin density operator: 
\begin{align}
\hat{S}^{a} =& \frac{1}{\mathcal{A}} \sum_{\kt} \sa , \nonumber \\ 
             &\sa = \sum_{CC'} S^{a}_{CC'} 
               \hat{a}^{\dag}_{C\kt} \hat{a}_{C\kt} , \nonumber \\ 
\hat{S}^{a}=& \frac{1}{\mathcal{A}} \sum_{CC'\kt} 
  S^{a}_{CC'}\hat{a}^{\dag}_{C\kt} \hat{a}_{C\kt} . \label{eq:areal_spin_density}
\end{align}

Substituting $\hat{n}$ and $\hat{S}^{a}$ in Eq.
\eqref{eq:FGR_1-2-I_contribution} we can write
\begin{subequations}\label{eq:FGR-n_and_Sa}
\begin{equation*}
\frac{d\ev*{\hat{n}}}{dt} =
\frac{d\ev*{\hat{n}}_{1}}{dt} +
\frac{d\ev*{\hat{n}}_{2}}{dt} +
\frac{d\ev*{\hat{n}}_{I}}{dt} ,
\end{equation*}
\begin{equation*}
\frac{d\ev*{\hat{S}^{a}}}{dt} =
\frac{d\ev*{\hat{S}^{a}}_{1}}{dt} +
\frac{d\ev*{\hat{S}^{a}}_{2}}{dt} +
\frac{d\ev*{\hat{S}^{a}}_{I}}{dt} .
\end{equation*}
\end{subequations}


\subsubsection{One-photon carrier and spin injection derivation}

The one-photon contribution to the carrier and the spin population injection can
be written as
\begin{align}
\frac{d\ev*{\hat{n}}^{e}_{1}}{dt} &=
\xi^{bc}_{1}(2\omega)E^{b*}(2\omega)E^{c}(2\omega) 
= \frac{1}{\mathcal{A}}\sum_{\kt}\xi^{bc}_{1}(2\omega;\kt) 
E^{b*}(2\omega)E^{c}(2\omega) ,
\label{eq:2w-carrier_population} \\
\frac{d\ev*{\hat{S}^{a}}^{e}_{1}}{dt} &=
\zeta^{abc}_{1}(2\omega)E^{b*}(2\omega)E^{c}(2\omega) 
= \frac{1}{\mathcal{A}}\sum_{\kt}\zeta^{abc}_{1}(2\omega;\kt)
E^{b*}(2\omega)E^{c}(2\omega) .
\label{eq:2w-spin_population}
\end{align}

Dropping the two-photon contribution, $\Omega^{II}_{CV}(\kt)$, in Eq. 
\eqref{eq:Omegas-complete} we define $\Omega_{CV}{\kt} \equiv
\Omega^{I}_{CV}(\kt)$. Then we can calculate the one-photon contribution for the
carrier and spin population injection using Eqns. \eqref{eq:e_contribution} and
\eqref{eq:1_electron_contribution} as follows:

Working with the operators we have 
\begin{align}
\mel*{C'\kt}{\hat{n}}{C\kt}
&= \frac{1}{\mathcal{A}} \sum_{C\kt} 
\mel*{C'\kt}{\hat{a}^{\dag}_{C\kt} \hat{a}_{C\kt}}{C\kt} , \nonumber \\
&= \frac{1}{\mathcal{A}} \sum_{C\kt} \delta_{CC'} . \label{eq:op_n}
\end{align}
\begin{align}
\mel*{C'\kt}{\hat{S}^{a}}{C\kt}
&= \frac{1}{\mathcal{A}} \sum_{C\kt} 
\mel*{C'\kt}{S^{a}_{C'C}\hat{a}^{\dag}_{C\kt} \hat{a}_{C\kt}}{C\kt} , \nonumber \\
&= \frac{1}{\mathcal{A}} \sum_{C\kt} 
S^{a}_{C'C}\mel*{C'\kt}{\hat{a}^{\dag}_{C\kt} \hat{a}_{C\kt}}{C\kt}
= \frac{1}{\mathcal{A}} \sum_{C\kt} S^{a}_{C'C}\delta_{CC'} \nonumber \\
&= \frac{1}{\mathcal{A}} \sum_{C\kt} S^{a}_{CC'} . \label{eq:op_S}
\end{align}

Working with the expressions of Eq. \eqref{eq:1_electron_contribution}
we have: 
\begin{align*}\label{eq:omegaICV_omegaI*VpV}
\Omega_{CV}{\kt} &= \frac{ie}{\hbar}
\frac{\mathbf{v}_{CV}(\kt) \cdot \mathbf{E}(2\omega)}{2\omega} , \\
\Omega^{*}_{C'V}{\kt} &= \frac{-ie}{\hbar}
\frac{\mathbf{v}^{*}_{C'V}(\kt) \cdot \mathbf{E}^{*}(2\omega)}{2\omega} .
\end{align*}
\begin{equation}\label{eq:dot_product}
\mathbf{v}_{CV}(\kt) \cdot \mathbf{E}(2\omega)
= \sum_{\ell} v^{\ell}_{CV}(\kt) E^{\ell}(2\omega) \equiv v^{\ell}_{CV}(\kt)
E^{\ell}(2\omega) ,
\end{equation}
we can write 
\begin{equation}\label{eq:omegaI*CpV_omegaICV_product}
[\Omega^{I*}_{C'V}(\kt) \Omega^{I}_{CV}(\kt)] =
\left( \frac{e}{\hbar} \right)^{2} 
\frac{v^{b*}_{C'V}(\kt)E^{b*}(2\omega)v^{c}_{CV}(\kt)E^{c}(2\omega)}{(2\omega)^{2}} .
\end{equation}

Then, using the expressions from Eq. \eqref{eq:op_n} to
\eqref{eq:omegaI*CpV_omegaICV_product} in Eq. 
\eqref{eq:1_electron_contribution} we can write
\begin{align}
\frac{d\ev*{\hat{n}}^{e}_{1}}{dt} &= 
\frac{2\pi e^{2}}{\mathcal{A}\hbar^{2}} \sum_{CV\kt}
\frac{\delta_{CC'}v^{b*}_{C'V}(\kt)E^{b*}(2\omega)v^{c}_{CV}(\kt)
E^{c}(2\omega)}{(2\omega)^{2}} \dw , \nonumber \\
&= \frac{2\pi e^{2}}{\mathcal{A}\hbar^{2}} \sum_{CV\kt}
\frac{v^{b*}_{CV}(\kt)E^{b*}(2\omega)v^{c}_{CV}(\kt)
E^{c}(2\omega)}{(2\omega)^{2}} \dw . \label{eq:2w_n-final}
\end{align}
\begin{equation}\label{2w_eq:S-final}
\frac{d\ev*{\hat{S}^{a}}^{e}_{1}}{dt} = 
\frac{2\pi e^{2}}{\mathcal{A}\hbar^{2}} \sum_{CC'V\kt} 
\frac{S^{a}_{C'C}v^{b*}_{C'V}(\kt)E^{b*}(2\omega)v^{c}_{CV}(\kt)E^{c}(2\omega)}
{(2\omega)^{2}}\dw
\end{equation}
and so, from Eq. \eqref{eq:2w-carrier_population} we have 

\begin{equation}\label{eq:1_carrier_tensor}
\xi^{bc}_{1}(2\omega;\kt) = \frac{2\pi e^{2}}{\hbar^{2}} 
\sum_{CV} \frac{v^{b*}_{CV}(\kt)v^{c}_{CV}(\kt)}{(2\omega)^{2}} \dw
\end{equation}

\begin{equation}\label{eq:1_spin_tensor}
\zeta^{bc}_{1}(2\omega;\kt) = \frac{2\pi e^{2}}{\hbar^{2}} 
\sum_{CC'V} \frac{S^{a}_{C'C}v^{b*}_{CV}(\kt)v^{c}_{CV}(\kt)}{(2\omega)^{2}} \dw
\end{equation}

{\color{blue} 
% El par de ecuaciones \eqref{eq:2w_n-final} y \eqref{eq:1_carrier_tensor}
% est\'an escritas como aparece el el art\'iculo pero tengo duda si deber\'ian
% escribirse as\'i:
% \begin{align*}
% \frac{d\ev*{\hat{n}}^{e}_{1}}{dt} 
% &= \frac{2\pi e^{2}}{\mathcal{A}\hbar^{2}} \sum_{{\color{red}CC'}V\kt}
% \frac{v^{b*}_{CV}(\kt)E^{b*}(2\omega){\color{red}v^{c}_{C'V}}(\kt)
% E^{c}(2\omega)}{(2\omega)^{2}} \delta(2\omega-\omega_{{\color{red}C'}V}(\kt)) .
% \end{align*}
% \begin{equation*}
% \xi^{bc}_{1}(2\omega;\kt) = \frac{2\pi e^{2}}{\hbar^{2}} 
% \sum_{{\color{red}CC'}V} \frac{v^{b*}_{CV}(\kt){\color{red}v^{c}_{C'V}}(\kt)}{(2\omega)^{2}} 
% \delta(2\omega-\omega_{{\color{red}C'}V}(\kt)) \nonumber
% \end{equation*}

El par de  ecuaciones \eqref{2w_eq:S-final}
y\eqref{eq:1_spin_tensor} son las que aparecen en el art\'iculo, pero no s\'e si 
deberian ser las siguientes:

\begin{align*}
\frac{d\ev*{\hat{S}^{a}}^{e}_{1}}{dt} &= 
\frac{2\pi e^{2}}{\mathcal{A}\hbar^{2}} \sum_{CC'V\kt} 
\frac{{\color{red}S^{a}_{CC'}}v^{b*}_{C'V}(\kt)E^{b*}(2\omega)v^{c}_{CV}(\kt)E^{c}(2\omega)}
{(2\omega)^{2}}\dw
\end{align*}
\begin{equation*}
\zeta^{bc}_{1}(2\omega;\kt) = \frac{2\pi e^{2}}{\hbar^{2}} 
\sum_{CC'V} \frac{{\color{red}S^{a}_{C'C}}v^{b*}_{CV}(\kt)v^{c}_{CV}(\kt)}{(2\omega)^{2}} 
\dw
\end{equation*}

Podr\'ias revisar?
}

\subsubsection{Two-photon carrier and spin injection derivation}

The two-photon contribution to the carrier and the spin population injection can
be written as
\begin{equation}
\begin{split}
\frac{d\ev*{\hat{n}}^{e}_{2}}{dt} &=
\xi^{bcdf}_{2}(\omega)E^{b*}(\omega)E^{c*}(\omega)E^{d}(\omega)E^{f}(\omega) \\
&= \frac{1}{\mathcal{A}}\sum_{\kt}\xi^{bcdf}_{2}(\omega;\kt) 
E^{b*}(\omega)E^{c*}(\omega)E^{d}(\omega)E^{f}(\omega) ,
\label{eq:1w_carrier_population} 
\end{split}
\end{equation}
\begin{equation}
\begin{split}
\frac{d\ev*{\hat{S}^{a}}^{e}_{2}}{dt} &=
\zeta^{abcdf}_{2}(\omega)E^{b*}(\omega)E^{c*}(\omega)E^{d}(\omega)E^{f}(\omega) \\
&= \frac{1}{\mathcal{A}}\sum_{\kt}\zeta^{abcdf}_{2}(\omega;\kt)
E^{b*}(\omega)E^{c*}(\omega)E^{d}(\omega)E^{f}(\omega) .
\label{eq:1w_spin_population}
\end{split}
\end{equation}

Now, keeping only the two-photon contribution, $\Omega^{II}_{CV}(\kt)$, in Eq. 
\eqref{eq:Omegas-complete} we can calculate the two-photon contribution for the
carrier and spin population injection using Eqns. \eqref{eq:e_contribution} and
\eqref{eq:2_electron_contribution} as follows:

Working with the expressions of Eq.
\eqref{eq:2_electron_contribution} and using the relation of Eq.
\eqref{eq:dot_product} we have
\begin{align*}\label{eq:omegaICV_omegaI*CpV}
\Omega^{II}_{CV}(\kt) &= - \left(\frac{e}{\hbar\omega}\right)^{2}
\sum_{CN} \frac{[\mathbf{v}_{CN}(\kt) \cdot \mathbf{E}(\omega)]
[\mathbf{v}_{NV}(\kt)\cdot\mathbf{E}(\omega)]}{\omega_{CV}(\kt)+\omega_{VN}(kt)},\\
\Omega^{II*}_{C'V}(\kt) &= - \left(\frac{e}{\hbar\omega}\right)^{2}
\sum_{C'M} \frac{[\mathbf{v}^{*}_{C'M}(\kt) \cdot \mathbf{E}^{*}(\omega)]
[\mathbf{v}^{*}_{MV}(\kt)\cdot\mathbf{E}^{*}(\omega)]}{\omega_{C'V}(\kt)+\omega_{VM}(kt)}, \\
\end{align*}
\begin{equation}\label{eq:omegaII*CpV_omegaIICV_product}
\begin{split}
[\Omega^{II}_{CV}(\kt)\Omega^{II*}_{C'V}(\kt)] = \left(\frac{e}{\hbar\omega}\right)^{4} &
\sum_{CC'MN} \frac{v^{b*}_{C'M}(\kt) E^{b*}(\omega)v^{c*}_{MV}(\kt) E^{c*}(\omega)
v^{d}_{C'N}(\kt) E^{d}(\omega)v^{f}_{NV}(\kt) E^{f}(\omega)}
{[\omega_{C'V}(\kt)/2+\omega_{VM}(kt)][\omega_{CV}(\kt)/2+\omega_{VN}(kt)]}
\end{split}
\end{equation}
and using the previous expression and the relations of Eqns. \eqref{eq:op_n} and
\eqref{eq:op_S}, we have 
\begin{align}
\frac{d\ev*{\hat{n}}^{e}_2}{dt} = \frac{1}{\mathcal{A}} \left(\frac{e}{\hbar\omega}\right)^{4}&
\sum_{CC'VMN} \frac{\delta_{CC'} v^{b*}_{C'M}(\kt) E^{b*}(\omega)v^{c*}_{MV}(\kt) E^{c*}(\omega)
v^{d}_{C'N}(\kt) E^{d}(\omega)v^{f}_{NV}(\kt) E^{f}(\omega)}
{[\omega_{C'V}(\kt)/2+\omega_{VM}(kt)][\omega_{CV}(\kt)/2+\omega_{VN}(kt)]} \nonumber \\
& \times \dw \nonumber \\
\begin{split}
= \frac{1}{\mathcal{A}} \left(\frac{e}{\hbar\omega}\right)^{4} &
\sum_{CVMN} \frac{v^{b*}_{CM}(\kt) E^{b*}(\omega)v^{c*}_{MV}(\kt) E^{c*}(\omega)
v^{d}_{CN}(\kt) E^{d}(\omega)v^{f}_{NV}(\kt) E^{f}(\omega)}
{[\omega_{C'V}(\kt)/2+\omega_{VM}(kt)][\omega_{CV}(\kt)/2+\omega_{VN}(kt)]} \\
& \times \dw \label{eq:2w_n_final}
\end{split}
\end{align}
\begin{equation}\label{eq:2w_S_final}
\begin{split}
\frac{d\ev*{\hat{S}^{a}}^{e}_2}{dt}= \frac{1}{\mathcal{A}} \left(\frac{e}{\hbar\omega}\right)^{4} &
\sum_{CC'VMN} \frac{S^{a}_{CC'} v^{b*}_{C'M}(\kt) E^{b*}(\omega)v^{c*}_{MV}(\kt) E^{c*}(\omega)
v^{d}_{CN}(\kt) E^{d}(\omega)v^{f}_{NV}(\kt) E^{f}(\omega)}
{[\omega_{C'V}(\kt)/2+\omega_{VM}(kt)][\omega_{CV}(\kt)/2+\omega_{VN}(kt)]} \\
& \times \dw 
\end{split}
\end{equation}

{\color{blue} La ecuaci\'on \eqref{eq:2w_n_final} est\'a escrita tal cual
como aparece en el art\'iculo. Creo que deber\'ia estar escrita de alguna de las 
dos siguientes formas:
\begin{equation*}
\begin{split}
\frac{d\ev*{\hat{n}}^{e}_2}{dt} = \frac{1}{\mathcal{A}} \left(\frac{e}{\hbar\omega}\right)^{4} &
\sum_{{\color{red}CC'}VMN} \frac{v^{b*}_{CM}(\kt) E^{b*}(\omega)v^{c*}_{MV}(\kt) E^{c*}(\omega)
v^{d}_{CN}(\kt) E^{d}(\omega)v^{f}_{NV}(\kt) E^{f}(\omega)}
{[{\color{red}\omega_{C'V}}(\kt)/2+\omega_{VM}(kt)][\omega_{CV}(\kt)/2+\omega_{VN}(kt)]} \\
& \times \dw
\end{split}
\end{equation*}
o
\begin{equation*}
\begin{split}
\frac{d\ev*{\hat{n}}^{e}_2}{dt} = \frac{1}{\mathcal{A}} \left(\frac{e}{\hbar\omega}\right)^{4} &
\sum_{{\color{red}C}VMN} \frac{v^{b*}_{CM}(\kt) E^{b*}(\omega)v^{c*}_{MV}(\kt) E^{c*}(\omega)
v^{d}_{CN}(\kt) E^{d}(\omega)v^{f}_{NV}(\kt) E^{f}(\omega)}
{[{\color{red}\omega_{CV}}(\kt)/2+\omega_{VM}(kt)][\omega_{CV}(\kt)/2+\omega_{VN}(kt)]} \\
& \times \dw
\end{split}
\end{equation*}

La ecuaci\'on \eqref{eq:2w_S_final} tengo duda si deber\'ia estar escrita as\'i:
\begin{equation}
\begin{split}
\frac{d\ev*{\hat{S}^{a}}^{e}_2}{dt}= \frac{1}{\mathcal{A}} \left(\frac{e}{\hbar\omega}\right)^{4} &
\sum_{CC'VMN} \frac{S^{a}_{CC'} v^{b*}_{C'M}(\kt) E^{b*}(\omega)v^{c*}_{MV}(\kt) E^{c*}(\omega)
v^{d}_{CN}(\kt) E^{d}(\omega)v^{f}_{NV}(\kt) E^{f}(\omega)}
{[{\color{red}\omega_{CV}}(\kt)/2+\omega_{VM}(kt)][\omega_{CV}(\kt)/2 + \omega_{VN}(kt)]} \\
& \times \dw 
\end{split}
\end{equation}
}

and so, from Eqns. \eqref{eq:1w_carrier_population} and
begin\eqref{eq:1w_spin_population} we have 
\begin{equation}
\xi^{bcdf}_{2}(\omega;\kt) = \left(\frac{e}{\hbar\omega}\right)^{4}
\sum_{CVMN} \frac{v^{b*}_{CM}(\kt)v^{c*}_{MV}(\kt)
v^{d}_{CN}(\kt)v^{f}_{NV}(\kt)}
{[\omega_{C'V}(\kt)/2+\omega_{VM}(kt)][\omega_{CV}(\kt)/2+\omega_{VN}(kt)]} 
\dw,
\end{equation}
\begin{equation}\label{eq:2_spin_tensor}
\zeta^{abcdf}_{2}(\omega;\kt) = \frac{1}{\mathcal{A}} \left(\frac{e}{\hbar\omega}\right)^{4} 
\sum_{CC'VMN} \frac{S^{a}_{CC'} v^{b*}_{C'M}(\kt)v^{c*}_{MV}(\kt)
v^{d}_{CN}(\kt)v^{f}_{NV}(\kt)}
{[\omega_{C'V}(\kt)/2+\omega_{VM}(kt)][\omega_{CV}(\kt)/2+\omega_{VN}(kt)]} \times \dw.
\end{equation}

\subsubsection{One- and two-photon interference carrier and spin injection derivation}

The one-  and two-photon interference contribution to the carrier and the spin
population injection can be written as
\begin{equation}\label{eq:int_carrier_population}
\begin{split}
\frac{d\ev*{\hat{n}}^{e}_{I}}{dt} &=
\xi^{bc}_{I}(\omega;2\omega)E^{b*}(\omega)E^{c*}(\omega)E^{d}(2\omega) + c.c., \\
&= \frac{1}{\mathcal{A}}\sum_{\kt}\xi^{bc}_{I}(2\omega;\kt) 
E^{b*}(\omega)E^{c*}(\omega)E^{d}(2\omega) + c.c. ,
\end{split}
\end{equation}
\begin{equation}\label{eq:int_spin_population}
\begin{split}
\frac{d\ev*{\hat{S}^{a}}^{e}_{I}}{dt} &=
\zeta^{abc}_{I}(\omega;2\omega)E^{b*}(\omega)E^{c*}(\omega)E^{d}(2\omega) + c.c. ,\\
&= \frac{1}{\mathcal{A}}\sum_{\kt}\zeta^{abc}_{I}(2\omega;\kt)
E^{b*}(\omega)E^{c*}(\omega)E^{d}(2\omega) + c.c.
\end{split}
\end{equation}

Finally, keeping both the one- and two-photon
contribution, $\Omega^{II}_{CV}(\kt)$ and $\Omega^{II}_{CV}(\kt)$, in Eq.
\eqref{eq:Omegas-complete} we can calculate the contribution due to the
interference of one-photon at $2\omega$ and two-photons at $\omega$ frequencies
for the carrier and spin population injection using Eqns.
\eqref{eq:e_contribution} and \eqref{eq:electron_I_contribution} as follows:

Working with the expression of Eq. \eqref{eq:electron_I_contribution} and using
the relation of Eqns. \eqref{eq:Omegas-complete} and  \eqref{eq:dot_product} we
have
\begin{equation*}
\Omega^{I}_{CV}{\kt} = \frac{ie}{\hbar}
\frac{\mathbf{v}_{CV}(\kt) \cdot \mathbf{E}(2\omega)}{2\omega} ,
\end{equation*}
\begin{equation*}
\Omega^{II*}_{C'V}(\kt) = - \left(\frac{e}{\hbar\omega}\right)^{2}
\sum_{C'M} \frac{[\mathbf{v}^{*}_{C'M}(\kt) \cdot \mathbf{E}^{*}(\omega)]
[\mathbf{v}^{*}_{MV}(\kt)\cdot\mathbf{E}^{*}(\omega)]}{\omega_{C'V}(\kt)+\omega_{VM}(kt)}
\end{equation*}
\begin{equation}\label{eq:omegaII*CpV_omegaICV_product}
[\Omega^{II*}_{C'V}(\kt) \Omega^{I}_{CV}{\kt}] =  -i \pi \left( \frac{e}{\hbar\omega} 
\right)^{3} \sum_{CC'MV} \frac{v^{b*}_{C'M}(\kt) E^{b*}(\omega)
v^{c*}_{MV}(\kt) E^{c*}(\omega) v^{d}_{cv}(\kt) E^{d}(2\omega) }
{\omega_{C'V}(\kt)+\omega_{VM}(kt)}
\end{equation}
and so, using the previous expression and the relations of Eqns. \eqref{eq:op_n} and
\eqref{eq:op_S}, we have 
\begin{equation}
\begin{split}
\frac{d\ev*{\hat{n}}^{e}_{I}}{dt} =&
- \frac{i \pi}{\mathcal{A}} \left( \frac{e}{\hbar\omega} 
\right)^{3} \sum_{CC'MV\kt} \frac{\delta_{CC'}v^{b*}_{C'M}(\kt) E^{b*}(\omega)
v^{c*}_{MV}(\kt) E^{c*}(\omega) v^{d}_{cv}(\kt) E^{d}(2\omega) }
{\omega_{C'V}(\kt)+\omega_{VM}(kt)} \dw \\
=& - \frac{i \pi}{\mathcal{A}} \left( \frac{e}{\hbar\omega} 
\right)^{3} \sum_{CC'MV\kt} \frac{v^{b*}_{CM}(\kt) E^{b*}(\omega)
v^{c*}_{MV}(\kt) E^{c*}(\omega) v^{d}_{cv}(\kt) E^{d}(2\omega) }
{\omega_{C'V}(\kt)+\omega_{VM}(kt)} \dw
\end{split}
\end{equation}
\begin{equation}
\begin{split}
\frac{d\ev*{\hat{n}}^{e}_{I}}{dt} &=
- \frac{i \pi}{\mathcal{A}} \left( \frac{e}{\hbar\omega} 
\right)^{3} \sum_{CC'MV\kt} \frac{S^{a}_{CC'}v^{b*}_{C'M}(\kt) E^{b*}(\omega)
v^{c*}_{MV}(\kt) E^{c*}(\omega) v^{d}_{cv}(\kt) E^{d}(2\omega) }
{\omega_{C'V}(\kt)+\omega_{VM}(kt)} \dw
\end{split}
\end{equation}

Then, making a comparison with Eqns. \eqref{eq:int_carrier_population} and 
\eqref{eq:int_spin_population} we can write 
\begin{equation*}
\xi^{bc}_{I}(2\omega,\omega;\kt) =  - i \pi \left( \frac{e}{\hbar\omega} 
\right)^{3} \sum_{CC'MV} \frac{v^{b*}_{CM}(\kt)
v^{c*}_{MV}(\kt) v^{d}_{cv}(\kt) }
{\omega_{C'V}(\kt)+\omega_{VM}(kt)} \dw
\end{equation*}
\begin{equation*}
\zeta^{bc}_{I}(2\omega,\omega;\kt) = - i \pi \left( \frac{e}{\hbar\omega} 
\right)^{3} \sum_{CC'MV} \frac{S^{a}_{CC'}v^{b*}_{C'M}(\kt)
v^{c*}_{MV}(\kt) v^{d}_{cv}(\kt) }
{\omega_{C'V}(\kt)+\omega_{VM}(kt)} \dw
\end{equation*}
% % section coherent_control (end)

\subsection{Electrical and spin currents}
{\small Section III-C in paper. \\}

Areal electrical current density operator ($\hat{J}{a}$):
\begin{equation}
\begin{split}
\hat{J}^{a} =& \frac{1}{\mathcal{A}} \sum_{\kt} \hat{J}^{a}(\kt) , \\
& \hat{J}^{a}(\kt) = e \sum_{CC'} v^{a}_{CC'}(\kt) \hat{a}^{\dag}_{C'\kt}
\hat{a}_{C\kt} , \\
\hat{J}^{a} =& \frac{e}{\mathcal{A}} \sum_{CC'\kt} v^{a}_{CC'}(\kt) 
\hat{a}^{\dag}_{C'\kt}\hat{a}_{C\kt} .
\end{split}
\end{equation}


Areal spin current density operator ($\hat{K}^{ab}=\hat{v}^{a}\hat{S}^{b}$): 
\begin{equation}
\begin{split}
\hat{K}^{ab} =& \frac{1}{\mathcal{A}} \sum_{\kt} K^{ab}(\kt) , \\
& K^{ab}(\kt) = \sum_{CC'} K^{ab}_{CC'}(\kt)\hat{a}^{\dag}_{C'\kt} 
\hat{a}_{C\kt} , \\
\hat{K}^{ab} =& \frac{1}{\mathcal{A}} \sum_{CC'\kt} K^{ab}_{CC'}(\kt) 
\hat{a}^{\dag}_{C'\kt} \hat{a}_{C\kt} .
\end{split}
\end{equation}

Now, using Eq. \eqref{eq:FGR_1-2-I_contribution} we can write 
\begin{equation*}
\begin{split}
\frac{d\ev*{\hat{J}^{a}}}{dt} =& \frac{d\ev*{\hat{J}^{a}}_{1}}{dt} + 
\frac{d\ev*{\hat{J}^{a}}_{2}}{dt} + \frac{d\ev*{\hat{J}^{a}}_{I}}{dt}  \\
\frac{d\ev*{\hat{K}^{ab}}}{dt} =& \frac{d\ev*{\hat{K}^{ab}}_{1}}{dt} + 
\frac{d\ev*{\hat{K}^{ab}}_{2}}{dt} + \frac{d\ev*{\hat{K}^{ab}}_{I}}{dt}  
\end{split}
\end{equation*}

\subsubsection{One-photon electrical- and spin-current}

The one-photon electrical and spin current terms are given by
\begin{align}
\frac{d\ev*{\hat{J}^{a}}_{1}}{dt} =& \eta^{acd}_{1}(2\omega) E^{c*}(2\omega)
E^{d}(2\omega), \nonumber \\ 
& \eta^{acd}_{1}(2\omega) = \frac{1}{\mathcal{A}} \sum_{\kt} \eta^{acd}_{1}
(2\omega;\kt), \nonumber \\
\frac{d\ev*{\hat{J}^{a}}_{1}}{dt} =& \frac{1}{\mathcal{A}} \sum_{\kt} 
\eta^{acd}_{1} (2\omega;\kt) E^{c*}(2\omega) E^{d}(2\omega) \label{eq:J_1-2w}
\end{align}
\begin{align}
\frac{d\ev*{\hat{K}^{ab}}_{1}}{dt} =& \mu^{abcd}_{1}(2\omega)
E^{c*}(2\omega) E^{d}(2\omega), \nonumber \\ 
& \mu^{abcd}_{1}(2\omega) = \frac{1}{\mathcal{A}} \sum_{\kt} \mu^{abcd}_{1}
(2\omega;\kt) , \nonumber \\
\frac{d\ev*{\hat{K}^{ab}}_{1}}{dt} =& \frac{1}{\mathcal{A}} \sum_{\kt} 
\mu^{abcd}_{1}(2\omega;\kt) E^{c*}(2\omega) E^{d}(2\omega), \label{eq:mu_1-2w}
\end{align}

Making the development fort the one-photon $2\omega$ electrical current density
we have
\begin{equation*}
\begin{split}
\frac{d \ev*{\hat{J}^{a}}_{1}}{dt} =& 2\pi \sum_{CC'V\kt} \mel*{C'\kt}
{\hat{J}^{a}(\kt)}{C\kt} [\Omega^{I}_{CV}(\kt) \Omega^{I*}_{C'V}(\kt)] 
\delta(2\omega - \omega_{CV}(\kt)) , \\
\end{split}
\end{equation*}
where
\begin{equation}\label{eq:J-matelem}
\begin{split}
\mel*{C'\kt}{\hat{J}^{a}(\kt)}{C\kt} &= 
\sum_{CC'\kt} \mel*{C'\kt}
{\frac{e}{\mathcal{A}} v^{a}_{CC'}(\kt) \hat{a}^{\dag}_{C'\kt} \hat{a}_{C\kt}}
{C\kt} \\
&= \frac{e}{\mathcal{A}} \sum_{CC'\kt} v^{a}_{CC'}(\kt) \mel*{C'\kt} 
{\hat{a}^{\dag}_{C'\kt} \hat{a}_{C\kt}}
{C\kt} \\
&= \frac{e}{\mathcal{A}} \sum_{CC'\kt}
v^{a}_{CC'}(\kt) \delta_{C'C} = \frac{e}{\mathcal{A}} \sum_{CC'\kt}
v^{a}_{{\color{red}{C'C}}}(\kt) .
\end{split}
\end{equation}
Then, using Eq. \eqref{eq:omegaI*CpV_omegaICV_product} we can write
\begin{equation}
\begin{split}
\frac{d\ev*{\hat{J}^{a}}_{1}}{dt} =& \frac{2 \pi e^{3}}{\mathcal{A} \hbar^{2}}
\sum_{CC'V\kt} \frac{v^{a}{C'C}(\kt) v^{c*}_{CV}(\kt) E^{c*}(2\omega) 
v^{d}_{C'V}(\kt) E^{d}(2\omega)}{(2\omega)^{2}} \delta(2\omega - 
\omega_{CV}(\kt))
\end{split}
\end{equation}
and making a comparison with Eq. \eqref{eq:J_1-2w} we have that the electrical 
current density tensor is given by
\begin{equation}
\eta^{acd}_1(2\omega;\kt) = \frac{2 \pi e^{3}}{\hbar^{2}} \sum_{CC'V}
\frac{v^{a}_{C'C}(\kt) v^{c*}_{CV}(\kt) v^{d}_{C'V}(\kt) }{(2\omega)^{2}}
\delta(2\omega - \omega_{CV}(\kt))
\end{equation}

The one-photon $2\omega$ spin current density is given by
\begin{equation}
\frac{d\ev*{\hat{K}^{ab}}_{1}}{dt} = 2 \pi \sum_{CC'V\kt} \mel{C'\kt}
{\hat{K}^{ab}(\kt)}{C\kt} [\Omega^{I}_{CV}(\kt) \Omega^{I*}_{C'V}(\kt)] \delta 
(2\omega - \omega_{CV}(\kt))
\end{equation}
where
\begin{equation}\label{eq:K-matelem}
\begin{split}
\mel{C'\kt}{\hat{K}^{ab}(\kt)}{C\kt} &= \sum_{CC'\kt} \mel{C'\kt}
{\frac{1}{\mathcal{A}} K^{ab}_{CC'}(\kt) 
\hat{a}^{\dag}_{C'\kt} \hat{a}_{C\kt})}{C\kt}  \\
&= \frac{1}{\mathcal{A}} \sum_{CC'\kt} K^{ab}_{CC'}(\kt) \mel{C'\kt}
{\hat{a}^{\dag}_{C'\kt} \hat{a}_{C\kt})}{C\kt}  \\
&= \frac{1}{\mathcal{A}} \sum_{CC'\kt}
K^{ab}_{CC'}(\kt) \delta_{CC'} = \frac{1}{\mathcal{A}} \sum_{CC'\kt}
K^{ab}_{\color{red}{C'C}}(\kt) .
\end{split}
\end{equation}
Using the Eq. \eqref{eq:omegaI*CpV_omegaICV_product} with a minimal
modification in the superscripts we can write
\begin{equation}\label{eq:omegaI*CpV_omegaICV_product-2}
[\Omega^{I*}_{C'V}(\kt) \Omega^{I}_{CV}(\kt)] =
\left( \frac{e}{\hbar} \right)^{2} 
\frac{v^{c*}_{C'V}(\kt)E^{c*}(2\omega)v^{d}_{CV}(\kt)E^{d}(2\omega)}
{(2\omega)^{2}} .
\end{equation}
and then
\begin{equation}
\frac{d\ev*{\hat{K}^{ab}}_{1}}{dt} = \frac{2 \pi e^{2}}{\mathcal{A} \hbar^{2}}
\sum_{CC'V\kt} \frac{K^{ab}(\kt) v^{c*}_{CV}(\kt) E^{c*}(2\omega) 
v^{d}_{CV}(\kt) E^{d}(2\omega) }{(2\omega)^{2}} \delta ({2\omega - 
\omega_{CV}(\kt)}) .
\end{equation}
Making a comparison with Eq. \eqref{eq:mu_1-2w} we have that the spin 
current density tensor is given by
\begin{equation}
\mu^{abcd}_{1} (2\omega;\kt) = \frac{2 \pi e^{2}}{\hbar^{2}}
\sum_{CC'V} \frac{K^{ab}(\kt) v^{c*}_{CV}(\kt) v^{d}_{CV}(\kt)}
{(2\omega)^{2}} \delta ({2\omega - \omega_{CV}(\kt)})
\end{equation}

\subsubsection{Two-photon electrical- and spin-current}

The two-photon $\omega$  electrical- and spin-current terms are given by
\begin{align}
\frac{d\ev*{\hat{J}^{a}}_{2}}{dt} =& \eta^{acd}_{2}(\omega) E^{c*}(\omega)
E^{d*}(\omega) E^{f}(\omega) E^{g}(\omega), \nonumber \\ 
& \eta^{acd}_{2}(\omega) = \frac{1}{\mathcal{A}} \sum_{\kt} \eta^{acd}_{2}
(\omega;\kt), \nonumber \\
\frac{d\ev*{\hat{J}^{a}}_{2}}{dt} =& \frac{1}{\mathcal{A}} \sum_{\kt} 
\eta^{acd}_{2} (\omega;\kt) E^{c*}(\omega) E^{d*}(\omega) E^{f}(\omega) 
E^{g}(\omega)\label{eq:J_2-1w}
\end{align}
\begin{align}
\frac{d\ev*{\hat{K}^{ab}}_{2}}{dt} =& \mu^{abcd}_{2}(\omega)
E^{c*}(\omega) E^{d*}(\omega) E^{f}(\omega) E^{g}(\omega), \nonumber \\ 
& \mu^{abcd}_{2}(\omega) = \frac{1}{\mathcal{A}} \sum_{\kt} \mu^{abcd}_{2}
(\omega;\kt) , \nonumber \\
\frac{d\ev*{\hat{K}^{ab}}_{2}}{dt} =& \frac{1}{\mathcal{A}} \sum_{\kt} 
\mu^{abcd}_{2}(\omega;\kt) E^{c*}(\omega) E^{d*}(\omega) E^{f}(\omega) 
E^{g}(\omega), \label{eq:K_2-1w}
\end{align}

Making the development fort the two-photon $\omega$ electrical current density
we have
\begin{equation*}
\begin{split}
\frac{d \ev*{\hat{J}^{a}}_{2}}{dt} =& 2\pi \sum_{CC'V\kt} \mel*{C'\kt}
{\hat{J}^{a}(\kt)}{C\kt} [\Omega^{II}_{CV}(\kt) \Omega^{II*}_{C'V}(\kt)] 
\delta(2\omega - \omega_{CV}(\kt)) . \\
\end{split}
\end{equation*}
Then, using the relation of Eq. \eqref{eq:omegaII*CpV_omegaIICV_product} with a
minimal modification in the superscripts we have
\begin{equation}\label{eq:omegaII*CpV_omegaIICV_product-2}
\begin{split}
[\Omega^{II}_{CV}(\kt)\Omega^{II*}_{C'V}(\kt)] = \left(\frac{e}{\hbar \omega}
\right)^{4} &
\sum_{CC'MN} \frac{v^{c*}_{C'M}(\kt) E^{c*}(\omega)v^{d*}_{MV}(\kt) E^{d*}
(\omega)
v^{f}_{C'N}(\kt) E^{f}(\omega)v^{g}_{NV}(\kt) E^{g}(\omega)}
{[\omega_{C'V}(\kt)/2+\omega_{VM}(kt)][\omega_{CV}(\kt)/2+\omega_{VN}(kt)]}
\end{split}
\end{equation}
and using Eq. \eqref{eq:J-matelem} we can write
\begin{equation}
\begin{split}
\frac{d\ev*{\hat{J}^{a}}_{2}}{dt} = \frac{2 \pi e^{5}}{\mathcal{A} 
(\hbar \omega)^{4}} \sum_{CC'VMN\kt} & \frac{v^{a}_{\color{red}{CC'}}(\kt) 
v^{c*}_{C'M}(\kt)E^{c*}(\omega) v^{d*}_{MV}(\kt) E^{d*}(\omega) 
v^{f}_{CN}(\kt) E^{f}(\omega) v^{g}_{NV}(\kt) E^{g}(\omega) }
{[\omega_{C'V}(\kt)/2 + \omega_{VM}] [\omega_{CV}(\kt)/2 + \omega_{VN}]} \\
& \times \delta(2\omega - \omega_{CV}(\kt))
\end{split}
\end{equation}
and making a comparison with Eq. \eqref{eq:J_2-1w} we have that the two-photon,
$\omega$, electrical current density tensor is given by
\begin{equation}
\begin{split}
\eta^{acdfg}_{2}(\omega;\kt) = \frac{2 \pi e^{5}}{(\hbar \omega)^{4}} 
\sum_{CC'VMN} & \frac{v^{a}_{\color{red}{CC'}}(\kt) v^{c*}_{C'M}(\kt) 
v^{d*}_{MV}(\kt) v^{f}_{CV}(\kt) v^{g}_{NV}(\kt)} {[\omega_{C'V}(\kt)/2 + 
\omega_{VM}] [\omega_{CV}(\kt)/2 + \omega_{VN}]} \delta(2\omega - 
\omega_{CV}(\kt))
\end{split}
\end{equation}

Making the development fort the two-photon $\omega$ spin current density
we have 
\begin{equation*}
\begin{split}
\frac{d \ev*{\hat{K}^{ab}}_{1}}{dt} =& 2\pi \sum_{CC'V\kt} \mel*{C'\kt}
{\hat{K}^{ab}(\kt)}{C\kt} [\Omega^{II}_{CV}(\kt) \Omega^{II*}_{C'V}(\kt)] 
\delta(2\omega - \omega_{CV}(\kt)) . \\
\end{split}
\end{equation*}
Using the relation of Eqns. \eqref{eq:omegaII*CpV_omegaIICV_product-2}
and \eqref{eq:J-matelem} we can write
\begin{equation}
\begin{split}
\frac{d\ev*{\hat{K}^{ab}}_{2}}{dt} = \frac{2 \pi e^{4}}{\mathcal{A} 
(\hbar \omega)^{4}} \sum_{CC'VMN\kt} & \frac{K^{ab}_{\color{red}{C'C}}(\kt) 
v^{c*}_{C'M}(\kt)E^{c*}(\omega) v^{d*}_{MV}(\kt) E^{d*}(\omega) 
v^{f}_{CN}(\kt) E^{f}(\omega) v^{g}_{NV}(\kt) E^{g}(\omega) }
{[\omega_{C'V}(\kt)/2 + \omega_{VM}] [\omega_{CV}(\kt)/2 + \omega_{VN}]} \\
& \times \delta(2\omega - \omega_{CV}(\kt))
\end{split}
\end{equation}
and making a comparison with Eq. \eqref{eq:K_2-1w} we have that the tone-photon
$\omega$ spin current density tensor is given by
\begin{equation}
\begin{split}
  \mu^{abcdfg}_{2}(\omega;\kt) = \frac{2 \pi e^{4}}{(\hbar \omega)^{4}} 
\sum_{CC'VMN} & \frac{K^{ab}_{\color{red}{C'C}}(\kt) 
v^{c*}_{C'M}(\kt) v^{d*}_{MV}(\kt) v^{f}_{CN}(\kt) v^{g}_{NV}(\kt) }
{[\omega_{C'V}(\kt)/2 + \omega_{VM}] [\omega_{CV}(\kt)/2 + \omega_{VN}]} 
\delta(2\omega - \omega_{CV}(\kt))
\end{split}
\end{equation}

\subsubsection{One- and two-photon interference electrical and spin currents}

The one-photon ($2\omega$) and two-photon ($\omega$)  electrical- and spin-
current terms can be written as
\begin{align}
\frac{d\ev*{\hat{J}^{a}}_{I}}{dt} = & \eta^{acdf}_{I}(2\omega,\omega) 
E^{c*}(\omega) E^{d*}(\omega) E^{f}(2\omega) + c.c. \nonumber \\
& \eta^{acdf}_{I}(2\omega,\omega) = \frac{1}{\mathcal{A}} \sum_{\kt} 
\eta^{acdf}_{I}(2\omega,\omega;\kt) , \nonumber \\
\frac{d\ev*{\hat{J}^{a}}_{I}}{dt} = & \frac{1}{\mathcal{A}} \sum_{\kt} 
\eta^{acdf}_{I}(2\omega,\omega;\kt) E^{c*}(\omega)
E^{d*}(\omega) E^{f}(2\omega) + c.c. \label{eq:J_12-I}
\end{align}
\begin{align}
\frac{d\ev*{\hat{K}^{ab}}_{I}}{dt} = & \eta^{acdf}_{I}(2\omega,\omega) 
E^{c*}(\omega) E^{d*}(\omega) E^{f}(2\omega) + c.c. \nonumber \\
& \eta^{acdf}_{I}(2\omega,\omega) = \frac{1}{\mathcal{A}} \sum_{\kt} 
\eta^{acdf}_{I}(2\omega,\omega;\kt) , \nonumber \\
\frac{d\ev*{\hat{K}^{ab}}_{I}}{dt} = & \frac{1}{\mathcal{A}} \sum_{\kt} 
\eta^{acdf}_{I}(2\omega,\omega;\kt) E^{c*}(\omega)
E^{d*}(\omega) E^{f}(2\omega) + c.c. \label{eq:K_12-I}
\end{align}

Making the development for the one-photon, $2\omega$, and two-photon, $\omega$,
interference, for the electrical current density we have 
\begin{equation*}
\frac{d\ev*{\hat{J}^{a}}_{I}}{dt} = 2 \pi \sum_{CC'V\kt} \mel{C'\kt}
{\hat{J}^{a}(\kt)}{C\kt} [\Omega^{I}_{CV} \Omega^{II*}_{C'V} + c.c.]
\delta(2\omega - \omega_{CV}(\kt))
\end{equation*}
Then, using the relation of Eq. \eqref{eq:omegaII*CpV_omegaICV_product} with a
minimal modification in the superscripts we have
\begin{equation}\label{eq:omegaII*CpV_omegaICV_product-2}
[\Omega^{II*}_{C'V}(\kt) \Omega^{I}_{CV}{\kt}] =  -i \pi \left( \frac{e}
{\hbar\omega} \right)^{3} \sum_{CC'MV} \frac{v^{c*}_{C'M}(\kt) E^{c*}(\omega)
v^{d*}_{MV}(\kt) E^{d*}(\omega) v^{f}_{cv}(\kt) E^{f}(2\omega) }
{\omega_{C'V}(\kt)+\omega_{VN}(\kt)}
\end{equation}
and using Eq. \eqref{eq:J-matelem} we can write
\begin{equation*}
\begin{split}
\frac{d\ev*{\hat{J}^{a}}_{I}}{dt} = - \frac{i \pi e^{3}}{\mathcal{A}(\hbar 
\omega)^{3}}\sum_{CC'VN\kt} & \frac{v^{a}_{C'C}(\kt) v^{c*}_{C'N}(\kt) 
E^{c*}(\omega) v^{d*}_{NV}(\kt) E^{d*}(\omega) v^{f}_{CV}(\kt) E^{f}(2\omega) }
{\omega_{C'V}(\kt)/2 + \omega_{VN}(\kt)} \\
& \times \delta(2\omega - \omega_{CV}(\kt)) .
\end{split}
\end{equation*}
Making a comparison with Eq. \eqref{eq:J_12-I} we have that the current
injection tensor for the quantum mechanical interference between the one- and
two-photon absorption can be written as
\begin{equation}
\eta^{acdf}_{I}(2\omega,\omega;\kt) = - \frac{i \pi e^{3}}{(\hbar \omega)^{3}}
\sum_{CC'VN} \frac{v^{a}_{C'C}(\kt) v^{c*}_{C'N}(\kt) v^{d*}_{NV}(\kt) 
v^{f}_{CV}(\kt)}{\omega_{C'V}(\kt)/2 + \omega_{VN}(\kt)} \delta(2\omega - 
\omega_{CV}(\kt)) .
\end{equation}

Finally, making the development for the one-photon, $2\omega$, and two-photon,
$\omega$, interference for the spin current density we have
\begin{equation*}
\frac{d\ev*{\hat{K}^{ab}}_{I}}{dt} = 2 \pi \sum_{CC'V\kt} \mel{C'\kt}
{\hat{K}^{ab}(\kt)}{C\kt} [\Omega^{I}_{CV} \Omega^{II*}_{C'V} + c.c.]
\delta(2\omega - \omega_{CV}(\kt))
\end{equation*}
Then, using the relation of Eqns. \eqref{eq:omegaII*CpV_omegaICV_product-2}
and \eqref{eq:J-matelem} we can write
\begin{equation*}
\begin{split}
\frac{d\ev*{\hat{K}^{ab}}_{I}}{dt} = - \frac{i \pi e^{3}}{\mathcal{A}(\hbar 
\omega)^{3}}\sum_{CC'VN\kt} & \frac{K^{ab}_{C'C}(\kt) v^{c*}_{C'N}(\kt) 
E^{c*}(\omega) v^{d*}_{NV}(\kt) E^{d*}(\omega) v^{f}_{CV}(\kt) E^{f}(2\omega) }
{\omega_{C'V}(\kt)/2 + \omega_{VN}(\kt)} \\
& \times \delta(2\omega - \omega_{CV}(\kt)) .
\end{split}
\end{equation*}
Making a comparison with Eq. \eqref{eq:J_12-I} we have that the current
injection tensor for the quantum mechanical interference between the one- and
two-photon absorption can be written as
\begin{equation}
\mu^{abcdf}_{I}(2\omega,\omega;\kt) = - \frac{i \pi e^{3}}{(\hbar 
\omega)^{3}}\sum_{CC'VN} \frac{K^{ab}_{C'C}(\kt) v^{c*}_{C'N}(\kt) 
v^{d*}_{NV}(\kt) v^{f}_{CV}(\kt) }{\omega_{C'V}(\kt)/2 + \omega_{VN}(\kt)} 
\delta(2\omega - \omega_{CV}(\kt)) .
\end{equation}



\end{document}


quitar la sumatoria sobre \kt en los tensores finales y revisar de que son
función














