\documentclass{article}
\usepackage{amsmath}
\usepackage{fullpage}
\usepackage[]{geometry}
\usepackage{color}
\usepackage{graphicx}
\usepackage{physics}

\newcommand{\kt}{\mathbf{k}_{t}}
\newcommand{\Op}{\hat{\mathcal{\theta}}}
\newcommand{\n}{\hat{n}(\kt)}
\newcommand{\sa}{\hat{S}^{a}(\kt)}
\newcommand{\dw}{\delta(2\omega - \omega_{CV}(\kt))}


%--------------------Link colors and pdf info-----------------------%
\usepackage{hyperref}
\definecolor{links}{rgb}{0,0.2,0.6}
\hypersetup{colorlinks,breaklinks,urlcolor=links,linkcolor=links,pdftitle=Reinaldo A. Zapata Pe\~na - Curriculum Vitae (Complete),pdfauthor=Reinaldo A. Zapata Pe\~na}

\title{Optical Injection and Control of Spin and Electrical Currents}
\author{Reinaldo Arturo Zapata Pe\~na}
\date{}

\begin{document}
\maketitle

Math development from
\href{http://journals.aps.org/prb/abstract/10.1103/PhysRevB.68.165348}{PRB
\textbf{68}, 165348 (2003)}: All-optical control of spin and electrical currents
in quantum wells; Ali Najaime, R.D.R. Bath, J.E. Sipe.


\section{Quantum well states} % (fold)
\label{sec:quantum_well_states}

\subsection{Parameters and matrix elements}
{\small (Section I-C in paper.) \\}

Double degeneracy at each $\kt$. $\psi_{ns\kt}$ $\leftarrow$ spinor
wavefunctions: $n \equiv$ subband index; $s\equiv$ degeneracy index. Also $n$
and $m$ are used to refer to an arbitrary subband index with $s$ or $p$
degeneracy. $c$ is used for the conduction subbands and $v$ for the valence
subbands. $A$ corresponds to the area in the $xy$ plane.

At the end of the calculation $A \rightarrow \infty$ $\Rightarrow$
\begin{equation*}
\frac{1}{A} \sum_{\kt} \rightarrow \int \frac{d\kt}{(2 \pi)^{2}} 
\end{equation*}

Definitions:
\begin{align*}
v_{n,s;m,p}^{a}(\kt) = \langle ns\kt | v^{a} |
mp\kt \rangle & \leftarrow \text{velocity matrix elements}\\
S_{n,s;m,p}^{a}(\kt) = \langle ns\kt | S^{a} |
mp\kt \rangle & \leftarrow \text{spin matrix elements}\\
K_{n,s;m,p}^{ab}(\kt)= \langle ns\kt | v^{a}S^{b} |
mp\kt \rangle & \leftarrow \text{spin-velocity matrix elements}
\end{align*}
Superscripts denote Cartesian components.

IPA, time recersal invariance leads to the following properties of the matrix
elements:
\begin{align*}
v_{n,\overline{s};m,\overline{p}}^{a*}(-\kt) &=
 -e^{i \lambda_{nm}} v_{n,s;m,p}^{a}(\kt)\\
S_{c,\overline{s};c,\overline{p}}^{a*}(-\kt) &=
 -e^{i \lambda'_{nm}} S_{c,s;c,p}^{a}(\kt)\\
K_{c,s;c,p}^{ab*}(\kt)&= 
 e^{i \lambda''_{nm}} K_{c,\overline{s};c,\overline{p}}^{ab}(-\kt)
\end{align*}
where $\overline{s}$ ($\overline{p}$) designates the opposite degeneracy index
of $s$ ($p$) and $\lambda$, $\lambda'$, and $\lambda''$ are arbitrary phases.
% section quantum_well_states (end)

\newpage
\section{Coherent control} % (fold)
\label{sec:coherent_control}
{\small Section II in paper}

For some experimental configuration the injected spin currents have an
accompanying electrical current. One can also generate \emph{pure} spin currents
where a sorting of optically injected process gives rise to electrons
propagating two opposite spatial directions with opposite average spins. The
direction of the propagation of the injected currents can be controlled via the
relative phase of the optical fields.

\subsection{A. Injection process}
{\small (Section II-A in paper.) \\}

Unperturbed Hamiltonian IPA second quantization:

\begin{equation*}\label{eq:unperturbed_hamiltonian}
H_{0} = \sum_{C\kt} \hbar \omega_{C}(\kt) \hat{a}^{\dag}_{C\kt} \hat{a}^{}_{C\kt}  - 
        \sum_{V\kt} \hbar \omega_{V}(\kt) \hat{b}^{\dag}_{C\kt} \hat{b}^{}_{C\kt}.
\end{equation*}

In a presence of an applied vector potential $\mathbf{A}(t)$ we have the
interaction Hamiltonian as
\begin{equation*}\label{eq:interaction_hamiltonian}
H_{int}(t) = - \frac{e}{c} \mathbf{A}(t) \cdot \mathcal{V}(t),
\end{equation*}
where
\begin{align*}
\mathcal{V}(t)  
=& \sum_{CC'\kt} \hat{a}^{\dag}_{C \kt} \hat{a}       _{C'\kt} 
\mathbf{v}_{CC'}(\kt) e^{i\omega_{CC'}(\kt)t} \\
-& \sum_{VV'\kt} \hat{b}^{\dag}_{V'\kt} \hat{b}       _{V \kt} 
\mathbf{v}_{VV'}(\kt) e^{i\omega_{VV'}(\kt)t} \\
+& \sum_{CV \kt} \hat{a}^{\dag}_{C \kt} \hat{b}^{\dag}_{V \kt} 
\mathbf{v}_{CV }(\kt) e^{i\omega_{CV }(\kt)t} \\
+& \sum_{CV \kt} \hat{b}       _{V \kt} \hat{a}       _{C \kt} 
\mathbf{v}_{VC }(\kt) e^{i\omega_{VC }(\kt)t},
\end{align*}
where $\hat{a}^{\dag}_{C\kt} $ ($\hat{a}_{C\kt}$) is the creation (annihilation)
operator for an electron with conduction subbands and spin indices denoted by
$c$ and $\hat{b}^{\dag}_{V\kt} $ ($\hat{b}_{V\kt}$) is the corresponding
creation (annihilation) operator for a hole with valence subbands and spin
indices denoted by $V$; $\hbar\omega_{CV}(\kt)$ is the energy of the conduction
and valence subbands $C/V$ at $\kt$

Vector potential of the field:
\begin{align*}\label{eq:vector_potential}
\mathbf{A}(t)   &=\mathbf{A}(  \omega)e^{-i(\omega+i\tau)t} 
                + \mathbf{A}(- \omega)e^{ i(\omega-i\tau)t} \\
                +&\mathbf{A}( 2\omega)e^{-i(\omega+i\tau)t} 
                + \mathbf{A}(-2\omega)e^{ i(\omega-i\tau)t},
\end{align*}
where $\tau$ is a small positive number to turn on the system at $t=-\infty$ and
is set to zero at the end of the calculation.

Electric field and vector potential relation:
\begin{equation*}\label{eq:electricF-vPotential_rel}
\mathbf{E} = \frac{i\omega}{c}\mathbf{A}(\omega).  
\end{equation*}

The optical field causes transitions from the ground state of the system,
$\ket{0}$, to the two particle (electron + hole) state:
\begin{equation*}\label{eq:transition}
\ket{CV\kt} = \hat{a}^{\dag}_{C\kt} \hat{a}_{V\kt} \ket{0}.
\end{equation*}

Perturbative approximation:
\begin{equation*}\label{eq:perturbative_approx}
\ket{\psi} = C_{0}(t) \ket{0} + \sum_{CV\kt} C_{CV\kt}(t)\ket{CV\kt}.
\end{equation*}
Solving for $C_{CV\kt}$ to second order in an electric field allows us to
calculate the rate of change of the expectation value of any single-particle
operator $\Op$ using the Fermi's golden rule (FGR, equation 6 of
the paper):
\begin{equation}\label{eq:FGR_1-2-I_contribution}
\frac{d \expval*{\Op}    }{dt} =
\frac{d \expval*{\Op}_{1}}{dt} +
\frac{d \expval*{\Op}_{2}}{dt} +
\frac{d \expval*{\Op}_{I}}{dt} ,
\end{equation}
where the subscripts 1, 2, and $I$ denotes one-photon contribution $2\omega$
process, two-photon $\omega$ process, and interference between excitation at 
$\omega$ and $2\omega$, respectively.

Each of the three components of Eq. \eqref{eq:FGR_1-2-I_contribution} will
include contributions from the conductions subbands $d\expval*{\Op}^{e}/dt$ and
valence subbands $d\expval*{\Op}^{h}/dt$ (equation 7 of the paper):
\begin{subequations}\label{eq:e-h_contribution_complete}
\begin{equation}\label{eq:e-h_contribution}
\frac{d\expval*{\Op}}{dt} =
\frac{d\expval*{\Op}^{e}}{dt} + 
\frac{d\expval*{\Op}^{h}}{dt} ,
\end{equation}
\begin{equation}\label{eq:e_contribution}
\frac{d\expval*{\Op}^{e}}{dt} = 
2\pi \sum_{C(C')V\kt} 
\Omega^{*}_{C'V}(\kt) 
\mel{C'\kt}{\Op}{C\kt}
\Omega_{CV}(\kt)
\delta\left( 2\omega-\omega_{CV}(\kt) \right),
\end{equation}
\begin{equation}\label{eq:h_contribution}
\frac{d\expval*{\Op}^{h}}{dt} = 
2\pi \sum_{V(V')C\kt} 
\Omega_{CV}(\kt)
\mel{V\kt}{\Op}{V'\kt}
\Omega^{*}_{CV'}(\kt)
\delta\left( 2\omega-\omega_{CV}(\kt) \right).
\end{equation}
\end{subequations}
Here the convention is $C=cs$, $C'=cs'$, $V=vs$, $V'=vs'$; so, the states differ
only in the spin index and then
\begin{equation*}\label{eq:omega_c-cp}
\omega_{CV}(\kt) = \omega_{C'V}(\kt).
\end{equation*}

The notation $C(C')$ indicates that the common subband index $c$ and the two
spin indices, $s$ and $s'$ are to be summed over. 

The term $\Omega_{CV}({\kt})$ in Eq. \eqref{eq:e-h_contribution_complete} is
given by: that (equation 8 in paper):
\begin{subequations}\label{eq:Omegas-complete}
\begin{equation}\label{eq:Omega_I-II}
\Omega_{CV}({\kt}) = \Omega^{I}_{CV}(\kt) + \Omega^{II}_{CV}(\kt),
\end{equation}
\begin{equation}\label{eq:Omega_I}
\Omega^{I}_{CV}(\kt) = \frac{ie}{\hbar}
\frac{\mathbf{v}_{CV}(\kt) \cdot \mathbf{E}(2\omega)}{2\omega}
\end{equation}
\begin{equation}\label{eq:Omega_II}
\Omega^{II}_{CV}(\kt) = - \frac{e}{\hbar\omega}
\sum_{N} \frac{[\mathbf{v}_{CN}(\kt) \cdot \mathbf{E}(\omega)]
[\mathbf{v}_{NV}(\kt)\cdot\mathbf{E}(\omega)]}{\omega_{CV}(\kt)+\omega_{VN}(kt)},
\end{equation}
\end{subequations}
where $\Omega^{I}(\kt)$ and $\Omega^{II}(\kt)$ correspond to one- and two-photon
transitions, respectively.

The hole subband have higher effective mass than the conduction subbands and so
the current is dominated by electrons. Then, substituting Eq.
\eqref{eq:e_contribution} into Eq. \eqref{eq:FGR_1-2-I_contribution} we have
that (equation 9 in paper)
\begin{subequations}\label{eq:electron_12I_contribution-complete}
\begin{equation}\label{eq:electron_12I_contribution}
\frac{d\expval*{\Op}^{e}}{dt} = 
\frac{d\expval*{\Op}^{e}_{1}}{dt} + 
\frac{d\expval*{\Op}^{e}_{2}}{dt} + 
\frac{d\expval*{\Op}^{e}_{I}}{dt} , 
\end{equation}
\begin{align}
\frac{d\expval*{\Op}^{e}_{1}}{dt} &= 
2\pi \sum_{C(C')V\kt}^n
\mel*{C'\kt}{\Op}{C\kt}
[\Omega^{I}_{CV}(\kt)
\Omega^{I*}_{C'V}(\kt)]
\delta \left( 2\omega - \omega_{CV}(\kt) \right) , 
\label{eq:1_electron_contribution} \\
\frac{d\expval*{\Op}^{e}_{2}}{dt} &= 
2\pi \sum_{C(C')V\kt}^n
\mel*{C'\kt}{\Op}{C\kt}
[\Omega^{II}_{CV}(\kt)
\Omega^{II*}_{C'V}(\kt)]
\delta \left( 2\omega - \omega_{CV}(\kt) \right) , 
\label{eq:electron_2_contribution}\\
\frac{d\expval*{\Op}^{e}_{I}}{dt} &= 
2\pi \sum_{C(C')V\kt}^n
\mel*{C'\kt}{\Op}{C\kt}
[\Omega^{I}_{CV}(\kt)
\Omega^{II*}_{C'V}(\kt) + c.c.]
\delta \left( 2\omega - \omega_{CV}(\kt) \right) 
\label{eq:electron_I_contribution}.
\end{align}
\end{subequations}

In previous equations the expression $\Op$ in can be substituted for the
electrical and spin current operators. Although the contributions from the holes
can be similarly found.

\subsection{B. Carrier and spin population}
In all the following expressions literal superscripts denote Cartesian
components and if repeated are to be summed over.

Areal carrier density operator: 
\begin{align}\label{eq:areal_carrier_density}
\hat{n} =& \frac{1}{\mathcal{A}} \sum_{\kt} \n , \nonumber \\ 
         &\n = \sum_{C} \hat{a}^{\dag}_{C\kt} 
           \hat{a}^{\dag}_{C\kt} , \nonumber \\ 
\hat{n} =& \frac{1}{\mathcal{A}} \sum_{C\kt} 
  \hat{a}^{\dag}_{C\kt} \hat{a}^{\dag}_{C\kt} . 
\end{align}

Areal $a$ component of the spin density operator: 
\begin{align}\label{eq:areal_spin_density}
\hat{S}^{a} =& \frac{1}{\mathcal{A}} \sum_{\kt} \sa , \nonumber \\ 
             &\sa = \sum_{CC'} S^{a}_{CC'} 
               \hat{a}^{\dag}_{C\kt} \hat{a}^{\dag}_{C\kt} , \nonumber \\ 
\hat{S}^{a}=& \frac{1}{\mathcal{A}} \sum_{C\kt} 
  \hat{a}^{\dag}_{C\kt} \hat{a}^{\dag}_{C\kt} . 
\end{align}

Substituting $\hat{n}$ and $\hat{S}^{a}$ in Eq.
\eqref{eq:FGR_1-2-I_contribution} we can write
\begin{subequations}\label{eq:FGR-n_and_Sa}
\begin{equation*}
\frac{d\ev*{\hat{n}}}{dt} =
\frac{d\ev*{\hat{n}}_{1}}{dt} +
\frac{d\ev*{\hat{n}}_{2}}{dt} +
\frac{d\ev*{\hat{n}}_{I}}{dt} ,
\end{equation*}
\begin{equation*}
\frac{d\ev*{\hat{S}^{a}}}{dt} =
\frac{d\ev*{\hat{S}^{a}}_{1}}{dt} +
\frac{d\ev*{\hat{S}^{a}}_{2}}{dt} +
\frac{d\ev*{\hat{S}^{a}}_{I}}{dt} .
\end{equation*}
\end{subequations}


199two-photons contribution, $\Omega^{II}_{CV}(\kt)$, in Eq. 
\eqref{eq:Omegas-complete} we define $\Omega_{CV}{\kt} \equiv
\Omega^{I}_{CV}(\kt)$. Then we can calculate the one-photon contribution to the
carrier and spin population injection using Eqns. \eqref{eq:e_contribution} and
\eqref{eq:1_electron_contribution} as follows:

The one-photon contribution to the carrier and the spin population injection can
be written as
\begin{align}
\frac{d\ev*{\hat{n}}^{e}_{1}}{dt} &=
\xi^{bc}_{1}(2\omega)E^{b*}(2\omega)E^{c}(2\omega) 
= \frac{1}{\mathcal{A}}\sum_{\kt}\xi^{bc}_{1}(2\omega;\kt) 
E^{b*}(2\omega)E^{c}(2\omega)
\label{eq:carrier_population} \\
\frac{d\ev*{\hat{S}^{a}}^{e}_{1}}{dt} &=
\zeta^{abc}_{1}(2\omega)E^{b*}(2\omega)E^{c}(2\omega) 
= \frac{1}{\mathcal{A}}\sum_{\kt}\zeta^{abc}_{1}(2\omega;\kt)
E^{b*}(2\omega)E^{c}(2\omega)
\label{eq:spin_population}
\end{align}
where the second-rank tensor and the third-rank pseudotensor are given by


\subsubsection{Carrier and spin injection derivation}
Working with the expressions of Eq. \eqref{eq:1_electron_contribution}
we have: 
\begin{align*}\label{eq:omegaICV_omegaI*VpV}
\Omega_{CV}{\kt} &= \frac{ie}{\hbar}
\frac{\mathbf{v}_{CV}(\kt) \cdot \mathbf{E}(2\omega)}{2\omega} , \\
\Omega^{*}_{C'V}{\kt} &= \frac{-ie}{\hbar}
\frac{\mathbf{v}^{*}_{C'V}(\kt) \cdot \mathbf{E}^{*}(2\omega)}{2\omega} .
\end{align*}
Now using the relation
\begin{equation}\label{eq:dot_product}
\mathbf{v}_{CV}(\kt) \cdot \mathbf{E}(2\omega)
= \sum_{\ell} v^{\ell}_{CV}(\kt) E^{\ell}(2\omega) \equiv v^{\ell}_{CV}(\kt)
E^{\ell}(2\omega) ,
\end{equation}
we can write 
\begin{equation}\label{eq:omegaICV_omegaI*VpV_product}
[\Omega^{I}_{CV}(\kt) \Omega^{I*}_{C'V}(\kt)] =
\left( \frac{e}{\hbar} \right)^{2} 
\frac{v^{b*}_{C'V}(\kt)E^{b*}(2\omega)v^{c}_{CV}(\kt)E^{c}(2\omega)}{(2\omega)^{2}} .
\end{equation}

Working with the operators we have 
\begin{align}\label{eq:op_n}
\mel*{C'\kt}{\hat{n}}{C\kt} ,
&= \frac{1}{\mathcal{A}} \sum_{C\kt} 
\mel*{C'\kt}{\hat{a}^{\dag}_{C\kt} \hat{a}^{\dag}_{C\kt}}{C\kt} , \nonumber \\
&= \frac{1}{\mathcal{A}} \delta_{CC'} .
\end{align}
\begin{align}\label{eq:op_S}
\mel*{C'\kt}{\hat{S}^{a}}{C\kt} ,
&= \frac{1}{\mathcal{A}} \sum_{CC'\kt} \mel*{C'\kt}{\hat{S}^{a}(\kt)}{C\kt} \nonumber \\
&= \frac{1}{\mathcal{A}} \sum_{CC'\kt} S^{a}_{C'C} \mel*{C'\kt}{(\kt)\hat{a}^{\dag}_{C\kt} 
\hat{a}^{\dag}_{C\kt}}{C\kt} , \nonumber \\
&= \frac{1}{\mathcal{A}} \sum_{CC'\kt} S^{a}_{C'C} \delta_{CC'} = 
\frac{1}{\mathcal{A}} \sum_{CC'\kt} S^{a}_{CC'} .
\end{align}


Then, using the three equations above into Eq.
\eqref{eq:1_electron_contribution} we can write
\begin{align*}
\frac{d\ev*{\n}^{e}_{1}}{dt} &= 
\frac{2\pi e^{2}}{\mathcal{A}\hbar^{2}} \sum_{CV\kt}
\frac{\delta_{CC'}v^{b*}_{C'V}(\kt)E^{b*}(2\omega)v^{c}_{CV}(\kt)
E^{c}(2\omega)}{(2\omega)^{2}} \dw , \\
&= \frac{2\pi e^{2}}{\mathcal{A}\hbar^{2}} \sum_{CV\kt}
\frac{v^{b*}_{CV}(\kt)E^{b*}(2\omega)v^{c}_{CV}(\kt)
E^{c}(2\omega)}{(2\omega)^{2}} \dw .
\end{align*}
and so, from Eq. \eqref{eq:carrier_population} we have that 
\begin{equation}\label{eq:1_carrier_tensor}
\xi^{bc}_{1}(2\omega;\kt) = \frac{2\pi e^{2}}{\hbar^{2}} 
\sum_{CV} \frac{v^{b*}_{CV}(\kt)v^{c}_{CV}(\kt)}{(2\omega)^{2}} \dw \nonumber
\end{equation}



% section coherent_control (end)
\end{document}




















