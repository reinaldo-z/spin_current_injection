\documentclass{article}
\usepackage{amsmath}
\usepackage{fullpage}
\usepackage[]{geometry}
\usepackage{color}
\usepackage{graphicx}


\newcommand{\kt}{\mathbf{k}_{t}}


%--------------------Link colors and pdf info-----------------------%
\usepackage{hyperref}
\definecolor{links}{rgb}{0,0.2,0.6}
\hypersetup{colorlinks,breaklinks,urlcolor=links,linkcolor=links,pdftitle=Reinaldo A. Zapata Pe\~na - Curriculum Vitae (Complete),pdfauthor=Reinaldo A. Zapata Pe\~na}

\title{Optical Injection and Control of Spin and Electrical Currents}
\author{Reinaldo Arturo Zapata Pe\~na}
\date{}

\begin{document}
\maketitle

Math development from
\href{http://journals.aps.org/prb/abstract/10.1103/PhysRevB.68.165348}{PRB
\textbf{68}, 165348 (2003)}: All-optical control of spin and electrical currents
in quantum wells; Ali Najaime, R.D.R. Bath, J.E. Sipe.


\section{Quantum well states} % (fold)
\label{sec:quantum_well_states}

\subsection*{C. Parameters and matrix elements}

Double degeneracy at each $\kt$. $\psi_{ns\kt}$ $\leftarrow$ spinor
wavefunctions: $n \equiv$ subband index; $s\equiv$ degeneracy index. Also $n$
and $m$ are used to refer to an arbitrary subband index with $s$ or $p$
degeneracy. $c$ is used for the conduction subbands and $v$ for the valence
subbands. $A$ corresponds to the area in the $xy$ plane.

At the end of the calculation $A \rightarrow \infty$ $\Rightarrow$
\begin{equation*}
\frac{1}{A} \sum_{\kt} \rightarrow \int \frac{d\kt}{(2 \pi)^{2}} 
\end{equation*}

Definitions:
\begin{align*}
v_{n,s;m,p}^{a}(\kt) = \langle ns\kt | v^{a} |
mp\kt \rangle & \leftarrow \text{velocity matrix elements}\\
S_{n,s;m,p}^{a}(\kt) = \langle ns\kt | S^{a} |
mp\kt \rangle & \leftarrow \text{spin matrix elements}\\
K_{n,s;m,p}^{ab}(\kt)= \langle ns\kt | v^{a}S^{b} |
mp\kt \rangle & \leftarrow \text{spin-velocity matrix elements}
\end{align*}
Superscripts denote Cartesian components.

IPA, time recersal invariance leads to the following properties of the matrix
elements:
\begin{align*}
v_{n,\overline{s};m,\overline{p}}^{a*}(-\kt) =&
 -e^{i \lambda_{nm}} v_{n,s;m,p}^{a}(\kt)\\
S_{c,\overline{s};c,\overline{p}}^{a*}(-\kt) =&
 -e^{i \lambda'_{nm}} S_{c,s;c,p}^{a}(\kt)\\
K_{c,s;c,p}^{ab*}(\kt)=& 
 e^{i \lambda''_{nm}} K_{c,\overline{s};c,\overline{p}}^{ab}(-\kt)
\end{align*}
where $\overline{s}$ ($\overline{p}$) designates the opposite degeneracy index
of $s$ ($p$) and $\lambda$, $\lambda'$, and $\lambda''$ are arbitrary phases.
% section quantum_well_states (end)


\section{Coherent control} % (fold)
\label{sec:coherent_control}
For some experimental configuration the injected spin currents have an
accompanying electrical current. One can also generate \emph{pure} spin currents
where a sorting of optically injected process gives rise to electrons
propagating two opposite spatial directions with opposite average spins. The
direction of the propagation of the injected currents can be controlled via the
relative phase of the optical fields.

\subsection*{A. Injection process}

Unperturbed Hamiltonian IPA second quantization:

\begin{equation*}\label{eq:unperturbed_hamiltonian}
H_{0} = \sum_{C\kt} \hbar \omega_{C}(\kt) a^{\dag}_{C\kt} a^{}_{C\kt}  - 
        \sum_{V\kt} \hbar \omega_{V}(\kt) b^{\dag}_{C\kt} b^{}_{C\kt}.
\end{equation*}

In a presence of an applied vector potential $\mathbf{A}(t)$ we have the
interaction Hamiltonian as
\begin{equation*}\label{interaction_hamiltonian}
H_{int}(t) = - \frac{e}{c} \mathbf{A}(t) \cdot \mathcal{V}(t),
\end{equation*}
where
\begin{align*}
\mathcal{V}(t)  
=& \sum_{CC'\kt} a^{\dag}_{C \kt} a       _{C'\kt} \mathbf{v}_{CC'}(\kt) e^{i\omega_{CC'}(\kt)t} \\
-& \sum_{VV'\kt} b^{\dag}_{V'\kt} b       _{V \kt} \mathbf{v}_{VV'}(\kt) e^{i\omega_{VV'}(\kt)t} \\
+& \sum_{CV \kt} a^{\dag}_{C \kt} b^{\dag}_{V \kt} \mathbf{v}_{CV }(\kt) e^{i\omega_{CV }(\kt)t} \\
+& \sum_{CV \kt} b       _{V \kt} a       _{C \kt} \mathbf{v}_{VC }(\kt) e^{i\omega_{VC }(\kt)t},
\end{align*}
where $a^{\dag}_{C\kt} $ ($a_{C\kt}$) is the creation (annihilation) operator
for an electron with conduction subbands and spin indices denoted by $c$ and
$b^{\dag}_{V\kt} $ ($b_{V\kt}$) is the corresponding creation (annihilation)
operator for a hole with valence subbands and spin indices denoted by $V$;
$\hbar\omega_{CV}(\kt)$ is the energy of the conduction and valence subbands
$C/V$ at $\kt$

Vector potential of the field:
\begin{align*}
\mathbf{A}(t)   =&\mathbf{A}(  \omega)e^{-i(\omega+i\tau)t} 
                + \mathbf{A}(- \omega)e^{ i(\omega-i\tau)t} \\
                +&\mathbf{A}( 2\omega)e^{-i(\omega+i\tau)t} 
                + \mathbf{A}(-2\omega)e^{ i(\omega-i\tau)t},
\end{align*}
where $\tau$ is a small positive number to turn on the system at $t=-\infty$ and
is set to zero at the end of the calculation.

Electric field and vector potential relation:
\begin{equation*}
\mathbf{E} = \frac{i\omega}{c}\mathbf{A}(\omega).  
\end{equation*}

The optical field causes transitions from the ground state of the system 

% section coherent_control (end)
\end{document}






















